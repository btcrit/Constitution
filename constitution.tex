\documentclass{article}
\begin{document}
\tableofcontents{}
\pagebreak
\section*{Preamble}
The purpose of the Blockchain Technology Club at the Rochester Institute of Technology is to educate students about blockchain techology and understand its implications on society.  BTC-RIT holds weekly informative presentations open to all students at RIT.
\section{Officers}\label{Officers}
\subsection{Executive Board}
\subsubsection{President}\label{President}
The President is the leader of the Blockchain Technology Club and is responsible for leading the club and organizing meetings.  The President is also responsible for communication with the Center for Campus Life, Student Affairs, Student Government, and any other organization .  The President is also responsible for scheduling speakers at all weekly meetings.  The President is also responsible for fulfilling the duties of any unfilled position.
\subsubsection{Vice President}\label{Vice President}
The Vice President is responsible for assisting the President in their duties and taking the position of President should the President be unable to fulfill their duties (whether temporarily or permanently).  The Vice President is also responsible for all funds, budgeting and financial records.
\subsubsection{Secretary}
The Secretary is responsible for collecting attendance and taking minutes at all weekly meetings.  They are also responsible for maintaining the weekly newsletter to be sent out to all members, as well as reviewing and presenting any necessary amendments to the constitution each semester.  In the event that the Vice President and President are both unable to fulfill their duties, the Secretary shall fulfill the duties of their positions.
\subsection{Cabinet Positions}
\subsubsection{Poster Chairman}
The Poster Chairman is responsible for distributing posters about upcoming meetings, and placing them around campus.
\subsection{Committees and Delegations of Authority}
Any Exeuctive Board or Cabinet member may create a committee at any time to fulfill part of their duties.  They may also delegate authority to a club member (with that club member's consent).  However, the executive board or cabinet member is ultimately responsible for fulfilling their duties even if it is delegated to another person or committee.  Any delegation of authority must be announced at the next general meeting.
\subsection{Removal}
\subsubsection{Impeachment}
An accusation of impeachment may be brought to any Executive Board member by any club member in good standing.  The charges must be brought up by said member at the next club meeting.  A vote will be held the following club meeting, which requires a 2/3ds majority of members (as described in \ref{Voting}).
\subsubsection{Resignation}
If a Cabinet or Executive Board member decides to resign or is unable to complete their term, a letter of resignation must be submitted to the Executive Board.  This resignation is final.
\subsubsection{Removal of the President}
In the event of removal of the President, the Vice President may assume the position of President for the remainder of the term with the vote of a 2/3ds majority of members.  Only members who have attended at least half of that semester's meetings will be eligible to vote in this vote.
\subsubsection{Removal Process}
After the removal of an officer, standard election procedure as outlined in [section] will occur at the next meeting.
\subsubsection{Holding of Positions}
Any club member is allowed to hold only one cabinet or executive board position at any given time.  The exceptions to this are that the Vice President may act as the President as described in \ref{Vice President} and that the President may act as any unfilled position as described in \ref{President}
\subsubsection{No members}
In the event where there are no members of Executive Board (for example, if all members needed to resign), an acting president will be chosen at random from all cabinet members present, or, if none are present, all members with voting rights (excluding those at their first meeting).  If the acting president accepts the nomination and is confirmed by 2/3ds of members, they will preside over the nomination and election of the new Executive Board.  The acting president is not eligible to serve in any Executive Board position from the election.
\section{Member Rights and Responsibilities}
\subsection{Member Requirements}
In order to be a member of the Blockchain Technology Club, members must be students enrolled full- or part-time at RIT.  Non-members may attend meetings at the discretion of the President.
\subsection{Member Dues}
There are no dues associated with being a member of the Blockchain Technology Club.
\subsection{Removal and Expulsion}
Members who are being rude or disruptive may be removed from a meeting at the discretion of the President or acting President.  Members who are repeatedly rude or disruptive may be expelled from the club by a vote of 2/3ds of the Executive Board.
\subsection{Presenting}
Any member may, and are encouraged to, present at meetings.  When and if this presentation takes place is at the discretion of the President.
\subsection{Voting}\label{Voting}
\subsubsection{Blind Vote}
Any vote held by the members will be a blind vote, that is, no member except the Executive Board may know who voted for which option, and in no event may the Executive Board share who voted for which option.
\subsubsection{Eligibility}
Any club member who has attended at least one-half of meetings over the course of the semester is eligible to vote.  Voting rights during the first meeting are determined by voting rights during the previous semester.  Members who are being impeached may not vote in their own impeachment vote, and may not count said vote.
\subsection{Elections}
\subsubsection{Timing}
Elections will take place every semester, or as needed when there is a vacancy.  Nominations will take place during the second-to-last meeting of the semester, or the first meeting where the vacancy is known.  Elections will take place the following meeting, with results counted by the end of the meeting.
\subsubsection{Eligibility}
In order to be eligible for an executive board position for the following semester, the member must be in good standing and be eligible to vote in elections, as well as being enrolled full- or part-time during the semester that they will be serving.  Exceptions may be made for members who are not enrolled during the semester of elections but will be enrolled during the following semester by a 2/3ds vote of the Executive Board.
\subsubsection{Nominations}
Eligible members may nominate themselves or be nominated by another member.  The member may either accept or decline their candidacy.  Any member not present has 72 hours to contact the secretary with their decision on whether they wish to accept or decline their nomination.  Any nominee has 72 hours from the time of their nomination to drop out of the race.  Nominees may run for two positions, but upon being elected to a second position must immediately resign the first.  Nominees running for two positions may also drop out of the other race upon being elected to a position.  A special newsletter must be distributed 72 hours after nominations, containing the list of all nominees as well as all members eligible to vote (assuming they attend the meeting), as well as voting procedures.
\subsection{Election Order}
Elections shall take place in the reverse order that they appear in section \ref{Officers}.  If a nominee is elected to a second position, a new election occurs immediately for the position that they resigned with no speeches.
\subsubsection{Speeches}
Each nominee must give an up-to-2-minute speech immediately preceeding elections, followed by up to 3 minutes of questions from members.  Nominees not present must notify the Secretary within 72 hours following their nomination (or as soon as possible), who will ask all members for questions. These nominees must present a speech to be read by the Presdient (which may not exceed 2 minutes in length), including their answers to any questions submitted by email.
\subsubsection{Voting}
Voting for elections will be done by a Preferential Ballot Vote immediately following speeches.  Absentee ballots may be completed prior to the election, and must be counted with standard ballots.  An option of "No Confidence" is also available on the ballot - in the case that this option is selected on a ballot it becomes the sole vote of the ballot.  In the event that "No Confidence" reaches a plurality of votes (that is, more than any individual candidate) the election is declared invalid and nominations begin the next week.  In the event that no candidate reaches a majority vote, the candidate with the fewest votes is eliminated and their votes go to the next choice of the individual - the process repeats until one candidate reaches majority or only one candidate remains.
\subsubsection{Voting method}
Voting may occur by paper ballot, or by any secure method as the Presdient may decide.
\section{Constitution}
\subsection{Consitutional Amendments}
An amendment to the constitution may be submitted by any member with voting rights.  This amendment must be distributed in the next email newsletter, and voted on at the next meeting.  A 2/3ds majority is required for any amendment.
\subsection{Initial Constitution Approval}
This Constitution must be approved by a vote of 2/3ds of members who would hold voting rights under it.
\end{document}
